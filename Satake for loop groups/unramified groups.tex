\section{The case of unramified groups}
    \subsection{Affine Grassmannians}
        \subsubsection{Loop groups}
            \begin{definition}[Jet spaces] \label{def: jet_spaces}
                If $X$ is a scheme over some commutative ring $k$ then we shall denote by $X[t]/t^n$ its associated \textbf{$n^{th}$ jet space}, which is the presheaf of sets on $\Sch_{/\Spec k}$ given by $S \mapsto X\left(S \x_{\Spec k} k[t]/t^n\right)$.
            \end{definition}
            \begin{definition}[Loop spaces] \label{def: loop_spaces}
                If $X$ is a scheme over a commutative ring $k$ then its associated \textbf{positive loop space} $X[\![t]\!]$ shall be the presheaf on $\Sch_{/\Spec k}$ given by $S \mapsto X\left(S \x_{\Spec k} \Spf k[\![t]\!]\right)$. Furthermore, the \textbf{loop space} associated to $X$, which we denote by $X(\!(t)\!)$, shall be the presheaf on $\Sch_{/\Spec k}$ given by $S \mapsto X\left(S \x_{\Spec k} \Spec k(\!(t)\!)\right)$.
            \end{definition}
            \begin{remark}
                It is easy to see that $X[\![t]\!] \cong \underset{n \in \N}{\lim} X[t]/t^{n + 1}$ for all $k$-schemes $X$, with the $\N$-cofiltered limit here being taken in the presheaf topos over the small fpqc site of $\Spec k$. As such, should $X$ is be of finite type over $\Spec k$, the associated positive loop space $X[\![t]\!]$ shall be of pro-finite type over $\Spec k$: in particular, this means that positive loop spaces of algebraic groups are of pro-finite type.
            \end{remark}
            The following is a technical result, necessary for the establishment of the affine Grassmannian for connected reductive groups $G$ as the fpqc quotient $G(\!(t)\!)/G[\![t]\!]$.
            \begin{proposition}[Loop spaces are fpqc sheaves] \label{prop: loop_spaces_are_fpqc_sheaves}
                Let $X$ be a scheme over finite type over a field $k$. Then, both $X[\![t]\!]$ and $X(\!(t)\!)$ will satisfy fpqc descent. 
            \end{proposition}
                \begin{proof}
                        
                \end{proof}
            
            \begin{definition}[The affine Grassmannian for $\GL_n$] \label{def: the_affine_grassmannian_for_GLn}
                Let $k$ be a commutative ring. The affine Grassmannian associated to the connected reductive group $k$-scheme, denoted by $\Gr_{\GL_n}$, is defined to be $\GL_n(\!(t)\!)/\GL_n[\![t]\!]$ with the quotient being taken inside the small fpqc topos of $\Spec k$.
            \end{definition}
            \begin{definition}[Integral lattices] \label{def: integral_lattices}
                Let $k$ be a commutative ring. A \textbf{$k$-integral lattice} of rank $n \geq 1$ is thus a $k[\![t]\!]$-submodule $\Lambda \subseteq k(\!(t)\!)^{\oplus n}$ such that $\Lambda\left[\frac1t\right] \cong k(\!(t)\!)^{\oplus n}$. 
            \end{definition}
            \begin{proposition}
                
            \end{proposition}
                \begin{proof}
                        
                \end{proof}
        
        \subsubsection{Lattices and affine Grassmannians}
        
        \subsubsection{Schubert cells}
        
    \subsection{Satake categories}
        \subsubsection{Construction and basic properties}
        
        \subsubsection{Universal local acyclicity}
        
        \subsubsection{Symmetric monoidal structures via convolutions}
        
        \subsubsection{Tannakian structures on Satake categories}
        
    \subsection{The Geometric Satake Equivalence for unramified split reductive groups}
        \subsubsection{The Satake Equivalence over geometric points}
        
        \subsubsection{Galois descent}